\documentclass[oneside,a4paper,10pts,article]{memoir}

\usepackage{palatino}
\usepackage{graphicx}
\usepackage{todonotes}

\title{Tradeable Certificates of Title on a blockchain \\
 {\normalfont\normalsize\scshape A case study on a trust-free vehicle registry}
}
\author{Martin Dybdal \texttt{dybber@dybber.dk}}
\date{\today}

\begin{document}

\maketitle

\begin{abstract}
  
\end{abstract}

\chapter{Preface}
This report was written as part of the ``Blockchain Summer School
2016'', which was held at the IT University of Copenhagen, August
2016. The summer school concluded with a 1-day hackathon, where we in
groups were assigned a case, that we should solve using blockchain
technology. Group-members: Felix Albert, Jacob Cholewa, Arun Prasad,
Benedikt Notheisen and Martin Dybdal.

\chapter{Introduction}
Blockchain technology is said to revolutionise businesses in domains
such as finance, logistics and administration, as documents,
contracts, certificates and so on, can be fully digitised. Blockchain
technology provides a decentralised database infrastructure which is
tamper-proof and trust-free. In essence, a blockchain is a
distributed, append-only digital ledger, which can be trusted, as
consensus on transactions are reached across all nodes in the network
and is made tamper-proof by the use of hashing algorithms. Blockchain
systems are called trust-free, as you do not need to trust any
individual parties, the trust is generated by contruction. A more
detailed introduction to blockchains is found in Section
\ref{sec:blockchain}.

In this project we have worked on a case arising in
public sector administration, from the Danish Tax Agency (SKAT). SKAT
is responsible for administering the registry of motorised vehicles in
Denmark. The Motor Registry is concerned with most aspects of a cars
life cycle, its initial registration on import, registration and
ownership taxation, insurance, police reports, MOT Tests, repairs and
change of ownership. The current system faces several problems related
to lack of documentation, imbalance of knowledge between parties and
imbalance of trust. It is SKATs expectation that a blockchain solution
might mitigate some of these issues.

The aspect of ownership change is a core problem in the digitisation
of \textit{Vehicle Titles} (dan. registreringsattester), as there is
an imbalance of knowledge and trust between the buyer and seller. We
have developed a small prototype system, which can represent a Vehicle
Title as a tradeable contract on the blockchain system
Ethereum. Though not implemented, we expect that our prototype can be
extended to also address the issues regarding lack of documentation,
such as documentation of repairs and accidents.

The rest of the report is structured as follows. In Section
\ref{sec:case} we will detail the major steps in Vehicle registration
and taxation in Denmark. In Section \ref{sec:trading} we describe the
core problems of trading contracts, and how it can be solved using a
blockchain. In Section \ref{sec:currency} we look at the problem of
representing non-crypto currencies, such as Danish Kroner or Euro's in
a blockchain. In Section \ref{sec:implementation} we describe our
Ethereum prototype and in Section \ref{sec:futurework} and
\ref{sec:conclusion} we describe future prospects and conclude.


\chapter{Blockchain}
\label{sec:blockchain}
Blockchain technology originates from the technology behind the
Bitcoin crypto-currency \cite{bitcoin}, where it is used to avoid the
problem of double-spending. Previously, double-spending was hindered
by using a trusted third party, which verified and timestamped all
transactions.

A blockchain is a series of timestamped data blocks, where each block
in the series contains a hash of the previous block. These hashes
serves the purpose of linking the blocks into chain, such that no
block in the chain can be tampered with, without also updating all
blocks following it. Furthermore, blockchain systems also employ
mechanisms to ensure that only one linear chain of blocks are allowed,
and that all parties agree on the same course of events. This makes
double-spending impossible, without requiring the need of a third
party. How consensus is reached depends on the particular blockchain
system.

To illustrate, we will use the next section to explain the system used
in the Bitcoin protocol, in the sections following we will discuss
various alternatives on how consensus is reached.

\section{The Bitcoin protocol}
 \cite{beck2016blockchain}

\subsection{Proof-of-Work}
In the original Bitcoin system, consensus on a single transaction
history is reached by ...



\section{Checkpointing}

\section{Proof-of-Stake}
Alternatives to proof of work has been suggested, one such alternative
is called Proof-of-Stake (PoS), originating from the
Bitcoin-alternative called PPCoin \cite{king2012ppcoin}. 

Nothing at stake

Proof-of-stake can not be trustless. Members of genesis block
  allocation can always create an alternate fork. Long-range attacks.
https://blog.ethereum.org/2014/07/05/stake/

\section{Proof-of-Space}

\section{Inherent limitations}

\begin{itemize}
\item https://download.wpsoftware.net/bitcoin/old-pos.pdf
\item   ``Impossibility of Distributed Consensus with One Faulty Process''
\item Byzantine Generals' Problem
\item Partition tolerancy and CAP Theorem
\end{itemize}




\chapter{Case description}
\label{sec:case}
Registration of vehicles, collecting vehicle taxes, and interacting
with other stakeholders, such as police, insurance companies and
transport authorities, requires a lot of administration, and a lot of
things can go wrong in process.

SKAT provided us with thorough case description, as well as answering
our questions regarding the Danish vehicle registration-process during
the summer school.

\section{The life cycle of a vehicle}
During the life cycle of vehicle, it goes through various owners and
activities, and involves several different parties, such as car
importers, police, insurance companies, MOT Testing companies and
transport authorities. When a car is imported into the country, it is
registered as owned by the importer, which sells it to a specific
dealer. The dealer then sells it to a private owner or a corporation,
and the car can be re-sold multiple times. See Figure \ref{fig:lifecycle}.

A crucial aspect is thus the change of ownership, which is registered
in the Danish Motor Registry. Especially the current process enforced
when trading used cars is problematic, as there is an imbalance of
knowledge and trust between the seller and the buyer. The buyer might
not know the full history of owners, accidents or repairs, and as the
Vehicle Title is currently on paper, it might be forged or the car
might be stolen. On the other hand, the seller has to put trust in the
buyer, that he will in fact re-register the car in his own
name. Currently, if the buyer does not immediately re-register the
car, the sellers insurance might still be covering damages and the
buyer could use the criminal activities, without being recorded as the
owner of the car.

This means that activities such as MOT Test, repairs and accident
reports are crucial for the buyer, and for the seller the
re-registration should be a part of the shift of ownership.

\begin{figure}
  \centering
  \includegraphics[width=\textwidth]{lifecycle.png}
  \caption{Vehicle life cycle}
  \label{fig:lifecycle}
\end{figure}

\section{Vehicle taxation in Denmark}
Taxation-wise there are several places where the Tax Agency needs to
be involved. When a car is imported, it has to be registered and a
registration tax has to be paid, it does not matter whether the car is
new or used. When a car shifts hands, the Tax Agency needs to know, as
they collect a yearly ownership tax for car owners. Also, several fees
can be levied as part of the process, for example a re-registration
fee. It is also the Tax Agency's job to make sure that all registered
cars are insured by their owner.

\section{Core challenges}
As outlined in the description above, the core challenges are:

\begin{itemize}
\item Giving buyer knowledge of vehicle history, which he can trust
\item Giving seller trust, that the buyer will not commit fraud, and
  use the vehicle in the sellers name.
\item Ensuring that ownership taxes are collected from the rightful
  owner of the vehicle
\end{itemize}


\chapter{Trading contracts}
\label{sec:trading}

Now let us look at how such blockchain systems can be used the issues
of the Danish Motor Registry. As described in Section \ref{sec:case},
the main underlying challenge is the transfer of ownership between a
seller and a buyer. This ownership is represented by a contract, a so
called Certificate of Title or in the case of a vehicles, a Vehicle
Title. Trading a vehicle, thus really means trading the Vehicle Title,
stating who is the rightful owner. In this section we will describe
the current process of such a trade, and we will describe how a
blockchain solution, can mitigate most of the current problems.

\todo{describe how a contract is traded today}

\todo{describe how a contract might be traded using blockchain}


This can be generalized, and instead talking about the specific case
of trading a vehicle, we can use the same framework for trading any
contract.


\chapter{Representing non-crypto currencies}
\label{sec:currency}
Describe that we assume a central bank will be willing to issue a
token contract, such that owners of Tokens on the blockchain can
convert such tokens back to actual danish kroner.

\chapter{Implementation}
\label{sec:implementation}

\section{Ethereum}
\cite{buterin2013ethereum, wood2014ethereum}

\section{Experience using Ethereum}
Describe how horrible the development environment is, how bad the
documentation is, and how terrible the error messages are.

Mention that Ethereum currently is good for prototyping, nothing else. 

Eventually a better language than Solidity may be introduced, but
currently not a good platform for any serious projects.

Mention Certified contract languages, such as the work by Tom Hvitved
or Patrick Bahr.

\chapter{User experience design}
NemID

Describe process suggested in Mockup-app from SKAT.

\chapter{Future work}
\label{sec:futurework}

\begin{itemize}
\item Proof of work, proof of stake, proof of space?
\item Can a system built now, be migrated to another system?
\item How much should be stored off the blockchain?
\item Prospect of using a more formaly defined contract specification language
\end{itemize}

\chapter{Conclusion}
\label{sec:conclusion}

\bibliographystyle{plain}
\bibliography{bibliography}



\end{document}